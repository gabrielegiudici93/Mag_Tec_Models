\documentclass[11pt,a4paper]{article}
\usepackage[margin=2.5cm]{geometry}
\usepackage{amsmath, amssymb}
\usepackage{booktabs}
\usepackage{hyperref}
\usepackage{siunitx}
\usepackage{enumitem}

\hypersetup{
    colorlinks=true,
    urlcolor=blue,
    linkcolor=blue,
    citecolor=blue
}

\title{Models Overview\\\large MagTec Tactile Skin Characterisation}
\author{MagTec Team}
\date{\today}

\begin{document}
\maketitle

\begin{abstract}
This document provides a brief overview of all machine learning models used in the MagTec tactile skin characterisation pipeline, including their purpose, training methodology, and operational principles. Both on-robot and simulation models are covered.
\end{abstract}

\tableofcontents
\newpage

\section{Model Types Overview}

The MagTec characterisation pipeline employs three main types of machine learning models, all based on Random Forest algorithms:

\begin{enumerate}[leftmargin=1.5em]
    \item \textbf{Force Mapping Regressors}: Predict normal force \(F_z\) from magnetic field features
    \item \textbf{Press Location Classifiers}: Classify which spatial location was pressed
    \item \textbf{Stretch Classifiers}: Predict the skin elongation level (0\%, 10\%, or 20\%)
\end{enumerate}

Additionally, a \textbf{Gated Pipeline} combines stretch and location classification in a two-stage approach.

\section{Force Mapping Regressors}

\subsection{Purpose}
Predict the normal force component \(F_z\) (in Newtons) from the magnetic field sensor readings.

\textbf{Note:} Currently, only \(F_z\) (normal/vertical force) is predicted. The tangential force components \(F_x\) and \(F_y\) are not estimated, as the vertical indentation protocol primarily generates normal forces. The FT sensor provides \((F_x, F_y, F_z, T_x, T_y, T_z)\), but only \(F_z\) is used as the target variable for training.

\subsection{Input Features}
\begin{itemize}[leftmargin=1.5em]
    \item \textbf{Full array}: 45 features (15 sensors \(\times\) 3 magnetic channels: \(B_x, B_y, B_z\))
    \item \textbf{Ablation study}: 9 features (3 central sensors \(\times\) 3 channels: sensors 7, 8, 9)
\end{itemize}

The magnetic field data is flattened from a \(15 \times 3\) matrix into a single 45-length vector:
\[
    \mathbf{s} = \left[\begin{array}{cccc}
        S_{1x} & S_{1y} & S_{1z} & \dotsc S_{15z}
    \end{array}\right]^T
\]

\subsection{Training Methodology}
\begin{itemize}[leftmargin=1.5em]
    \item \textbf{Per-stretch models}: One Random Forest regressor per stretch level (0\%, 10\%, 20\%)
    \item \textbf{Combined model}: One regressor trained on pooled data across all stretches
    \item \textbf{Data split}: 70/30 train/test split
    \item \textbf{Hyperparameters}: 200 trees, unlimited depth, bootstrap enabled
    \item \textbf{Target}: \(F_z\) force component only (continuous value in Newtons). Note: \(F_x\) and \(F_y\) are not predicted.
\end{itemize}

\subsection{How It Works}
The model learns the mapping:
\[
    \hat{F}_z = f_s(\mathbf{s})
\]
where \(f_s\) is a Random Forest function that takes the magnetic field vector \(\mathbf{s}\) and outputs a predicted force value. The model captures the relationship between magnet displacement (encoded in magnetic field changes) and applied force.

\subsection{Performance Metrics}
\begin{itemize}[leftmargin=1.5em]
    \item \textbf{RMSE} (Root Mean Square Error): Measures prediction accuracy
    \item \textbf{STD} (Standard Deviation of residuals): Measures precision
    \item \textbf{Force Resolution} (\(\Delta F_{\text{min}}\)): Minimum distinguishable force difference
\end{itemize}

\section{Press Location Classifiers}

\subsection{Purpose}
Classify which spatial location on the skin was pressed, given the magnetic field sensor readings.

\subsection{Input Features}
Same as force regressors: 45 features (full array) or 9 features (central subset for ablation study).

\subsection{Training Methodology}
\begin{itemize}[leftmargin=1.5em]
    \item \textbf{Per-stretch classifiers}: One Random Forest classifier per stretch level
    \item \textbf{Combined classifier}: One classifier trained on pooled data (stretch-invariant)
    \item \textbf{Data split}: 70/30 train/test split, stratified by stretch
    \item \textbf{Hyperparameters}: 200 trees, unlimited depth, bootstrap enabled
    \item \textbf{Target}: Discrete location labels
        \begin{itemize}
            \item \textbf{Single-point dataset}: \(\{\text{center}, \text{nw}, \text{ne}, \text{se}, \text{sw}\}\) (5 classes)
            \item \textbf{Multi-point dataset}: \(\{\text{center}, \text{nw}, \text{ne}, \text{se}, \text{sw}, 4, 5, 6, 7, 9, 10, 11, 12\}\) (13 classes)
        \end{itemize}
\end{itemize}

\subsection{How It Works}
The per-stretch classifier learns the mapping:
\[
    \hat{p}_s: \mathbf{s} \mapsto \{c_1, c_2, \ldots, c_{n_s}\}
\]
where \(c_i\) are the press location classes available for stretch level \(s\). The classifier identifies spatial patterns in the magnetic field that correspond to different contact locations.

The combined classifier learns stretch-invariant patterns:
\[
    \hat{p}_{\text{comb}}: \mathbf{s} \mapsto \{c_1, c_2, \ldots, c_n\}
\]
where the class set includes all press locations seen across all stretch levels.

\subsection{Performance Metrics}
\begin{itemize}[leftmargin=1.5em]
    \item \textbf{Accuracy}: Overall classification accuracy
    \item \textbf{Precision}: Per-class precision (\(\text{TP}_c / (\text{TP}_c + \text{FP}_c)\))
    \item \textbf{Recall}: Per-class recall (\(\text{TP}_c / (\text{TP}_c + \text{FN}_c)\))
    \item \textbf{F1-score}: Harmonic mean of precision and recall
\end{itemize}

\section{Stretch Classifier}

\subsection{Purpose}
Predict the skin elongation level (0\%, 10\%, or 20\%) from the magnetic field sensor readings.

\subsection{Input Features}
Same as other models: 45 features (full array) or 9 features (central subset).

\subsection{Training Methodology}
\begin{itemize}[leftmargin=1.5em]
    \item \textbf{Single classifier}: One Random Forest classifier trained on pooled dataset
    \item \textbf{Data split}: 70/30 train/test split, stratified by stretch
    \item \textbf{Hyperparameters}: 250 trees, unlimited depth, bootstrap enabled
    \item \textbf{Target}: Stretch level labels \(\{0\%, 10\%, 20\%\}\) (3 classes)
\end{itemize}

\subsection{How It Works}
The classifier learns the mapping:
\[
    \hat{l}: \mathbf{s} \mapsto \{0\%, 10\%, 20\%\}
\]
The model detects how skin elongation changes the magnetic field patterns. As the skin stretches, the magnet grid geometry changes, producing distinct signatures in the magnetic field measurements.

\subsection{Performance Metrics}
\begin{itemize}[leftmargin=1.5em]
    \item \textbf{Accuracy}: Overall classification accuracy
    \item \textbf{Per-class metrics}: Precision, recall, and F1-score for each stretch level
\end{itemize}

\section{Gated Pipeline}

\subsection{Purpose}
A two-stage classification system that combines stretch detection with location classification for robust press location prediction.

\subsection{How It Works}
The gated pipeline uses a hierarchical approach:

\begin{enumerate}[leftmargin=1.5em]
    \item \textbf{Stage 1}: Predict stretch level using the stretch classifier
        \[
            \hat{l}(\mathbf{s}) \in \{0\%, 10\%, 20\%\}
        \]
    \item \textbf{Stage 2}: Route to the corresponding per-stretch press location classifier
        \[
            \hat{p} = \begin{cases}
                \hat{p}_{\hat{l}(\mathbf{s})}(\mathbf{s}) & \text{if per-stretch model exists and is trained}, \\
                \hat{p}_{\text{comb}}(\mathbf{s}) & \text{otherwise (fallback to combined classifier)}.
            \end{cases}
        \]
\end{enumerate}

\subsection{Advantages}
\begin{itemize}[leftmargin=1.5em]
    \item \textbf{Robustness}: Falls back to combined classifier if per-stretch model unavailable
    \item \textbf{Specialization}: Uses stretch-specific models when available for better accuracy
    \item \textbf{Flexibility}: Handles unseen stretch levels gracefully
\end{itemize}

\section{Simulation Models}

\subsection{Overview}
The simulation models use the same architecture and training methodology as the on-robot models, but are trained on synthetic data generated by finite element simulations.

\subsection{Data Source}
\begin{itemize}[leftmargin=1.5em]
    \item \textbf{Input format}: HDF5 files containing:
        \begin{itemize}
            \item \texttt{MagneticField} \((N, 15, 3)\): Raw \(B_x, B_y, B_z\) readings
            \item \texttt{IdenterPosition} \((N, 3)\): Tool pose in metres
            \item \texttt{forcesTest} \((N, 3)\): Cartesian forces (Fx, Fy, Fz)
        \end{itemize}
    \item \textbf{Position labels}: Derived by rounding indenter \(x/y\) coordinates to 0.1~mm
    \item \textbf{Stretch labels}: Inferred from filenames or dataset attributes
\end{itemize}

\subsection{Model Types}
The simulation pipeline trains the same model types:
\begin{itemize}[leftmargin=1.5em]
    \item Per-stretch force regressors (if \texttt{forcesTest} is present)
    \item Per-stretch position classifiers
    \item Pooled stretch classifier
    \item Pooled position classifier
\end{itemize}

\subsection{Training Process}
\begin{enumerate}[leftmargin=1.5em]
    \item Convert simulation HDF5 files to match robot data structure
    \item Extract magnetic field features (flatten to 45-length vector)
    \item Derive position labels from \texttt{IdenterPosition} coordinates
    \item Train models using identical pipeline as on-robot data
    \item Export models and metrics to \texttt{data/Imported/<run-label>/models/}
\end{enumerate}

\subsection{Key Differences from On-Robot Models}
\begin{itemize}[leftmargin=1.5em]
    \item \textbf{No physical FT sensor}: Uses \texttt{forcesTest} from simulation instead
    \item \textbf{Position derivation}: Labels derived from indenter coordinates rather than robot offsets
    \item \textbf{Data quality}: Synthetic signals may have different noise characteristics
    \item \textbf{Training pipeline}: Uses \texttt{train\_simulation\_positions.py} instead of \texttt{evaluate\_single\_point\_stretch.py}
\end{itemize}

\section{Training Pipeline}

\subsection{General Workflow}
\begin{enumerate}[leftmargin=1.5em]
    \item \textbf{Data Collection}: Robot presses at multiple locations/stretches → HDF5 files with press summaries
    \item \textbf{Feature Extraction}: Flatten \(15 \times 3\) magnetic channels → 45-length vector
    \item \textbf{Train/Test Split}: 70/30 stratified split (by stretch for classification)
    \item \textbf{Model Training}: Random Forest with specified hyperparameters
    \item \textbf{Evaluation}: Compute metrics (RMSE, accuracy, etc.) on test set
    \item \textbf{Export}: Save models as joblib files in \texttt{models/} subdirectory
    \item \textbf{Reporting}: Generate JSON metrics file with all performance indicators
\end{enumerate}

\subsection{Hyperparameters Summary}
All models use Random Forest with the following common settings:
\begin{itemize}[leftmargin=1.5em]
    \item \textbf{Force regressors}: 200 trees, unlimited depth
    \item \textbf{Location classifiers}: 200 trees, unlimited depth
    \item \textbf{Stretch classifier}: 250 trees, unlimited depth
    \item \textbf{Features per split}: \(\sqrt{d}\) (default)
    \item \textbf{Bootstrap}: Enabled
    \item \textbf{Random state}: 42 (for reproducibility)
\end{itemize}

\subsection{Model Storage}
\begin{itemize}[leftmargin=1.5em]
    \item \textbf{On-robot models}: \texttt{data/<run-label>/models/}
    \item \textbf{Simulation models}: \texttt{data/Imported/<run-label>/models/}
    \item \textbf{Format}: Joblib files (Python serialization)
    \item \textbf{Metrics}: JSON files with performance metrics
\end{itemize}

\section{Model Relationships}

All models share the same input feature space (magnetic field measurements) but differ in their:
\begin{itemize}[leftmargin=1.5em]
    \item \textbf{Target variable}: Continuous (force) vs. discrete (location, stretch)
    \item \textbf{Training data subset}: Per-stretch vs. pooled across stretches
    \item \textbf{Output space}: Regression (force value) vs. classification (class label)
\end{itemize}

The gated pipeline demonstrates how multiple models can be combined to create a more robust system that leverages both stretch-specific and general patterns in the data.

\section*{Summary}

The MagTec characterisation pipeline employs a comprehensive set of Random Forest models to:
\begin{itemize}[leftmargin=1.5em]
    \item Predict applied forces from magnetic field patterns
    \item Classify press locations with high accuracy
    \item Detect skin elongation states
    \item Combine models in a gated pipeline for robust operation
\end{itemize}

All models are trained using consistent methodologies and can be applied to both on-robot and simulation data, enabling validation and comparison across different data sources.

\end{document}

